\subsection{Introduction to Cosmology}
With Hubble's discovery of the expanding universe, there has been great efforts to understand this expansion and the evolution of this expansion (and of the universe as a whole).
A related idea, General Relativity has described the importance of the geometry of the universe.
There are, in general, three possible geometries:
\begin{itemize}
    \item A flat geometry which is equivalent to Euclidean space with zero curvature.
    \item An open geometry with constant negative curvature (Anti-de Sitter space).
    \item A Closed geometry with constant positive curvature (de Sitter space)
\end{itemize}
Given the relationship between curvature and the energy-momentum tensor given by the Einstein field equations, it seems reasonable to assume that any non-zero energy means the universe is not flat.
Through quantum field theory (QFT) the Casimir effect shows that the vacuum has non-zero energy density, and through astronomical observation of distant galaxies it can be seen that our universe is flat.
At first these two observations contradict each other. However, by introducing another term to the Einstein field equations, this discrepency can be resolved.
\[ G_{\mu\nu} - \underbrace{\Lambda g_{\mu\nu}}_{\text{new term}} = T_{\mu\nu} \]
The constant $\Lambda$ is called the \textit{cosmological constant} which can absorb the contributions from the vacuum energy, allowing for a flat universe and non-zero vacuum energy density.
A natural question to ask is `what is the source of the cosmological constant?'

\subsection{$\Lambda$CDM}

\subsection{The (Affine) Geodesic Equation}
Suppose we have a (semi-) Riemannian manifold $M$ with metric $g$ and tangent bundle $TM$. An \textit{affine connection} is a map
\[ \Gamma (TM) \times \Gamma (TM) \rightarrow \Gamma (TM) \]
\[ (X,Y) \rightarrow \nabla_X Y \]
That is, it parallel transports the vector field $Y$ along the connection $\nabla$ in the direction of vector field $X$. From this, we can write the affine geodesic equation for a path $\gamma(t)$
\[ \nabla_{\dot\gamma} \dot\gamma(t) = 0 \]
Thus, a geodesic is a path such that its tangent vector is parallel translated. Since we observe our world with coordinates, in physics it is often more instructive to work this out in a specific set of coordinates $x^\mu$. Thus this can be written as
\[ \ddot{\gamma}^\mu + \Gamma^{\mu}_{\rho\lambda}\dot{\gamma}^\rho \dot{\gamma}^\lambda  \]


\subsection{The FLRW Metric and Dark Energy}
In general, metrics allow one to attribute distances between points in a space as $d = g_{\mu\nu}x^\mu x^\nu$. 
The `flat' metric for relativity is the \textit{Minkowski metric} given by $\text{diag}(-1,1,1,1)$, however, a metric to describe the expanding universe is given by the \textit{FLRW metric} by $d^2 = t^2-a^2(t)s^2$ with $s^2$ the standard Euclidean distance in $\mathbb{R}^3$. 
What is immediately appearent from the FLRW metric is that spacial slices of the $d=4$ spacetime remain curvature free under the expansion of the universe describe by the scale factor $a^2(t)$.

When describing the structure of the universe, I will make a few assumptions (which hold up to small perturbations):
\begin{itemize}
    \item Homogeneity. The cosmology describing the universe does not depend on location.
    \item Isotropy. The cosmology describing the universe does not depend on location.
\end{itemize}
These two conditions for what is sometimes referred to as \textit{the cosmological principle}. In general, they don't hold on small scales, however averaging over a sufficiently large distance these assumptions give an accurate description. The isotropy condition means that the universe should have 0 net momentum, and by assumming the universe is smooth the energy-momentum tensor can be written
\begin{equation} T^\mu_\nu = \left(
\begin{array}{cccc}
    -\mathcal{E} & 0 & 0 & 0 \\
    0 & \mathcal{P} & 0 & 0 \\
    0 & 0 & \mathcal{P} & 0 \\
    0 & 0 & 0 & \mathcal{P}
\end{array}
\right)
\end{equation}
The usual conservation law holds
\begin{equation}
    \nabla_\mu T^{\mu}_{\nu} = 0 \Rightarrow \partial_t \mathcal{E} + \frac{\dot a}{a}(3\mathcal{E} + 3\mathcal{P}) = 0
\end{equation}
We can use the geodesic equation to examine how the energy of the massless particles evolves through time. 

