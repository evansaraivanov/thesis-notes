\documentclass{article}
\usepackage{geometry}
 \geometry{
 a4paper,
 total={170mm,257mm},
 left=20mm,
 top=20mm,
 }
\usepackage[utf8]{inputenc}
\usepackage{hyperref}
\usepackage{braket}
\usepackage{amsmath}
\usepackage{amssymb}
\usepackage{amsthm}
\usepackage{mathtools}
\usepackage{amsfonts}
\usepackage{physics}
\usepackage{tikz-cd}
\usepackage{musicography}
\usepackage{enumerate}
\usepackage{xfrac}
\usepackage[parfill]{parskip}
\usepackage{subcaption}
\usepackage{float}
\usepackage{pgfplots}
\pgfplotsset{compat=1.12}
\usepgfplotslibrary{fillbetween}

\newcommand{\cm}{\mathbb{C}}
\newcommand{\re}{\mathbb{R}}
\newcommand{\dvg}{\nabla\cdot}
\newcommand{\crl}{\nabla\times}
\newcommand{\proj}[1]{\ensuremath{{\ket{#1}\bra{#1}}}}
\newcommand{\vect}{\text{Vect}}
\newcommand{\smooth}{C^{\infty}}
\newcommand{\dif}{\text{d}}

\makeatletter
\newcommand*\bigcdot{\mathpalette\bigcdot@{.8}}
\newcommand*\bigcdot@[2]{\mathbin{\vcenter{\hbox{\scalebox{#2}{$\m@th#1\bullet$}}}}}
\makeatother
\newlength{\arrow}
\settowidth{\arrow}{\scriptsize$1000$}
\newcommand*{\myrightarrow}[1]{\xrightarrow{\mathmakebox[\arrow]{#1}}}
\newcommand*{\TakeFourierOrnament}[1]{{%
\fontencoding{U}\fontfamily{futs}\selectfont\char#1}}
\newcommand*{\danger}{\TakeFourierOrnament{66}}
\setlength{\jot}{1.5mm}

\theoremstyle{definition}
\newtheorem{thm}{Theorem}[section] % the main one
\newtheorem{lemma}[thm]{Lemma}
\newtheorem{joke}{Joke}
% other statement types

% for specifying a name
\theoremstyle{plain} % just in case the style had changed
\newcommand{\thistheoremname}{}
\newtheorem{genericthm}[thm]{\thistheoremname}
\newenvironment{namedthm}[1]
  {\renewcommand{\thistheoremname}{#1}%
   \begin{genericthm}}
  {\end{genericthm}}
  
\newtheorem{defn}{Definition}
\newtheorem{prop}{Proposition}
\newtheorem{rmk}{Remark}
\newtheorem{conj}{Conjecture}
%\newtheorem{lemma}{Lemma}
\newtheorem{cor}{Corollary}
\newtheorem{problem}{Problem}




\begin{document}

\begin{center}
	\huge{Research notes}
	
	\small{All here is public}
\end{center}

\section{Statistics}
\subsection{Introduction to Cosmology}
With Hubble's discovery of the expanding universe, there has been great efforts to understand this expansion and the evolution of this expansion (and of the universe as a whole).
A related idea, General Relativity has described the importance of the geometry of the universe.
There are, in general, three possible geometries:
\begin{itemize}
    \item A flat geometry which is equivalent to Euclidean space with zero curvature.
    \item An open geometry with constant negative curvature (Anti-de Sitter space).
    \item A Closed geometry with constant positive curvature (de Sitter space)
\end{itemize}
Given the relationship between curvature and the energy-momentum tensor given by the Einstein field equations, it seems reasonable to assume that any non-zero energy means the universe is not flat.
Through quantum field theory (QFT) the Casimir effect shows that the vacuum has non-zero energy density, and through astronomical observation of distant galaxies it can be seen that our universe is flat.
At first these two observations contradict each other. However, by introducing another term to the Einstein field equations, this discrepency can be resolved.
\[ G_{\mu\nu} - \underbrace{\Lambda g_{\mu\nu}}_{\text{new term}} = T_{\mu\nu} \]
The constant $\Lambda$ is called the \textit{cosmological constant} which can absorb the contributions from the vacuum energy, allowing for a flat universe and non-zero vacuum energy density.
A natural question to ask is `what is the source of the cosmological constant?'

\subsection{$\Lambda$CDM}

\subsection{The (Affine) Geodesic Equation}
Suppose we have a (semi-) Riemannian manifold $M$ with metric $g$ and tangent bundle $TM$. An \textit{affine connection} is a map
\[ \Gamma (TM) \times \Gamma (TM) \rightarrow \Gamma (TM) \]
\[ (X,Y) \rightarrow \nabla_X Y \]
That is, it parallel transports the vector field $Y$ along the connection $\nabla$ in the direction of vector field $X$. From this, we can write the affine geodesic equation for a path $\gamma(t)$
\[ \nabla_{\dot\gamma} \dot\gamma(t) = 0 \]
Thus, a geodesic is a path such that its tangent vector is parallel translated. Since we observe our world with coordinates, in physics it is often more instructive to work this out in a specific set of coordinates $x^\mu$. Thus this can be written as
\[ \ddot{\gamma}^\mu + \Gamma^{\mu}_{\rho\lambda}\dot{\gamma}^\rho \dot{\gamma}^\lambda  \]


\subsection{The FLRW Metric and Dark Energy}
In general, metrics allow one to attribute distances between points in a space as $d = g_{\mu\nu}x^\mu x^\nu$. 
The `flat' metric for relativity is the \textit{Minkowski metric} given by $\text{diag}(-1,1,1,1)$, however, a metric to describe the expanding universe is given by the \textit{FLRW metric} by $d^2 = t^2-a^2(t)s^2$ with $s^2$ the standard Euclidean distance in $\mathbb{R}^3$. 
What is immediately appearent from the FLRW metric is that spacial slices of the $d=4$ spacetime remain curvature free under the expansion of the universe describe by the scale factor $a^2(t)$.

When describing the structure of the universe, I will make a few assumptions (which hold up to small perturbations):
\begin{itemize}
    \item Homogeneity. The cosmology describing the universe does not depend on location.
    \item Isotropy. The cosmology describing the universe does not depend on location.
\end{itemize}
These two conditions for what is sometimes referred to as \textit{the cosmological principle}. In general, they don't hold on small scales, however averaging over a sufficiently large distance these assumptions give an accurate description. The isotropy condition means that the universe should have 0 net momentum, and by assumming the universe is smooth the energy-momentum tensor can be written
\begin{equation} T^\mu_\nu = \left(
\begin{array}{cccc}
    -\mathcal{E} & 0 & 0 & 0 \\
    0 & \mathcal{P} & 0 & 0 \\
    0 & 0 & \mathcal{P} & 0 \\
    0 & 0 & 0 & \mathcal{P}
\end{array}
\right)
\end{equation}
The usual conservation law holds
\begin{equation}
    \nabla_\mu T^{\mu}_{\nu} = 0 \Rightarrow \partial_t \mathcal{E} + \frac{\dot a}{a}(3\mathcal{E} + 3\mathcal{P}) = 0
\end{equation}
We can use the geodesic equation to examine how the energy of the massless particles evolves through time. 



\section{Computing}
\subsection{Markov Chain Monte Carlo (MCMC)} 
Interestingly, MCMC algorithms have heavy analogies with statistical mechanics which are useful to demonstrate the concept. To examine this, lets first define what a Markov Chain is.
\begin{defn}
A sequence $X_1,\hdots,X_n$ of random elements is a \textit{Markov Chain} if the conditional distribution $X_{n+1}$ depends only on $X_n$. The set in which $X_i$ take values is called the \textit{state space} of the chain.
\end{defn}


\subsection{The Metropolis-Hastings Algorithm}
Suppose we want to sample from a distribution $p(x)$. $p(x)$ can be high dimensional and is generally difficult to calculate (evidence is hard to compute since it requires integration over the entire parameter space). The goal is to use Markov Chains to sample from $p(x)$ without needing to compute the evidence. This will be represented as a path through state space until the chain reaches a stable point (stationary state).

We start with a proposal distribution $g(x_n)$. Sample from the proposal distribution to find the next state $x_{n+1}$ with probability $g(x_{n+1}|x_n)$. This transition from state $n$ to state $n+1$ must follow the \textit{detailed balance condition}
\[ p(x_n) g(x_{n+1}|x_n) A(x_n \rightarrow x_{n+1}) = p(x_{n+1}) g(x_n|x_{n+1}) A(x_{n+1} \rightarrow x_{n}) \]
where $A$ is an \textit{acceptance probability} which I will define more precisely later. Using Bayes' Theorem on $p(x)$, the evidence cancels out on each side, and thus the detailed balance condition can be simplified to only rely on the likelihood and prior of $p(x)$, which I will denote $\pi$ and $\mathcal{L}$.
\[ \pi(x_n)\mathcal{L}(x_n) g(x_{n+1}|x_n) A(x_n \rightarrow x_{n+1}) = \pi(x_{n+1}) \mathcal{L}(x_{n+1}) g(x_{n}|x_{n+1}) A(x_{n+1}\rightarrow x_{n}) \]
\[ \Rightarrow \frac{A(x_n \rightarrow x_{n+1})}{A(x_{n+1}\rightarrow x_{n})} = \frac{\pi(x_{n+1}) \mathcal{L}(x_{n+1}) g(x_{n}|x_{n+1})}{\pi(x_n)\mathcal{L}(x_n) g(x_{n+1}|x_n)} \equiv R_{n,n+1} \]
This allows us to define the acceptance probability as
\[ A(x_n \rightarrow x_{n+1}) = \min( 1, R_{n,n+1} ) \]
This probability is used to determine whether the chain moves to $x_{n+1}$ or stays at $x_n$. The chain converges when it reaches a stationary state.

There are a few properties that can be observed for this algorithm:
\begin{itemize}
    \item Having an asymmetrical proposal $g(x)$ can allow for faster convergence of the chain.
    \item The initial sampling may not accurately reflect samples for $p(x)$. This is regarded as the `burn-in' and is generally discarded from the samples.
    \item MCMC Sampling loses sampling power for multi-modal distributions. 
\end{itemize}

\subsection{Parameter Difference Distribution}
Define the posterior probability of parameters $\theta$ within model $\mathcal{M}$ given data $d$ as (Bayes' theorem)
\begin{equation}
    \mathcal{P}(\theta) := P(\theta|d,\mathcal{M}) = \frac{P(\theta|\mathcal{M})P(d|\theta,\mathcal{M})}{P(d|\mathcal{M})}
\end{equation}
The following notation will simplify this expression
\begin{equation*}
    \begin{split}
        \Pi(\theta) &= P(\theta|\mathcal{M}) \text{, Prior probability} \\
        \mathcal{L}(\theta) &= P(d|\theta,\mathcal{M}) \text{, the likelihood} \\
        \mathcal{E} &= P(d|\mathcal{M}) \text{, the evidence}
    \end{split}
\end{equation*}
\[ \mathcal{P}(\theta) := P(\theta|d,\mathcal{M}) = \frac{\Pi(\theta)\mathcal{L}}{\mathcal{E}} \]
These are (maybe?) understandable as the prior probability is the parameters given the model, the likelihood as the data given some parameters of the model, and the evidence as the data give a model. From now on, the model $\mathcal{M}$ will be implied.

Consider now two data sets $d_1,d_2$. They have a joint likelihood 
\[ \mathcal{L}(\theta) = P(d_1,d_2|\theta) \]
This \textbf{quantifies the likelihood of $d_1$ and $d_2$ coming from the same set of parameters of a given model.} Denote now $\mathcal{L}_1,\mathcal{P}_1$ as the marginalized likelihood and Posterior over $d_2$. (\textbf{Is there a procedure/algorithm to find the marginalized probability? Seems generally non-trivial}).

Now duplicate the parameter set to $\theta_1,\theta_2$ and perscibe a new joint likelihood $\mathcal{L}(\theta_1,\theta_2) = P(d_1,d_2|\theta_1,\theta_2)$. \textbf{This is generally a choice, but how can one make a good choice?} The choice however is not unique.

As such, the constraints are imposed:
\begin{enumerate}
    \item $\mathcal{L}(\theta_1=\theta,\theta_2=\theta) = \mathcal{L}(\theta)$, or in plain english, the joint likelihood if $\theta_1=\theta=\theta_2$ needs to coincide with the likelihood of $\theta$
    \item $P(d_1|\theta_1,\theta_2)=P(d_1|\theta_1)$ once marginalized over $d_2$. Marginalizing the likelihood over one of the data sets removes the dependancy on the corresponding parameter sets.
\end{enumerate}
 \textbf{This ensures the datasets are conditionally independent,}
\[ P(d_1,d_2|\theta) = P(d_1|\theta)P(d_2|\theta) \]
\textbf{Proof?}
Which means we can choose 
\[ \mathcal{L}(\theta_1,\theta_2) = \mathcal{L}_1(\theta_1)\mathcal{L}_2(\theta_2) \]
If we further assume the prior distribution can be factorized, the joint posterior is then
\[ \mathcal{P}(\theta_1,\theta_2) = \mathcal{L}(\theta_1,\theta_2)\Pi(\theta_1)\Pi(\theta_2) \]
Defining $\Delta\theta = \theta_1-\theta_2$, the parameter difference posterior is given by
\[ \mathcal{P}(\Delta\theta) = \int_{V_\Pi} \mathcal{P}(\theta,\theta-\Delta\theta)d\theta\]

\subsection{Quantifying Results}

Given some probability $P$ of a parameter shift, the following formula can give you the number of standard deviations if the probability shift comes from a gaussian distribution
\[ n_\sigma = \sqrt{2} \text{Erf}^{-1}(P) \]
I have a notebook using two unit gaussian priors separated by a distance $a$. This example can be computed analytically.
\begin{equation*}
    \begin{split}
	\mathcal{P}(\Delta \theta) &= \frac{1}{2\pi} \int\limits_{-\infty}^{\infty} e^{-\theta^2/2} e^{-{(\theta-\Delta\theta)}^2/2}  d\theta \\
				   &= \frac{1}{2\pi} \cdot \sqrt{\pi} e^{-{(\Delta\theta)}^2/4}\\
				   &= \frac{1}{\sqrt{4\pi}}e^{-{(\Delta\theta)}^2/4}\\
    \end{split}
\end{equation*}
The parameter difference posterior is a gaussian with standard deviation $\sqrt{2}$. The separation is fixed by $a$, hence the shift is $\mathcal{P}(a)$. Hence the shift probability is
\[ \Delta = \int\limits_{-a}^{a} e^{-{(\Delta\theta)}^2/4} d\Delta\theta \]
Lets use the example $a=2$. Then $n_\sigma = 2/\sqrt{2} = \sqrt{2}$. Using this we can work backwards to find $\Delta$ from a $z$-table to find $\Delta = 0.9207 - 0.0793 = 0.8414 $. 

\subsection{Normalizing Flows}
The method of normalizing flows (MAF) implemented here uses Masked Autoencoders (MADE) to construct the flow. Suppose we have an input to the flow $x_i$. The output of the map is $y_i= \mu(x_{1:i-1})+\sigma(x_{1:i-1})x_i$. The $\mu$ and $\sigma$ are found using neural networks which recieve masked inputs $x_{1:i-1}=(x_1,\ldots,x_{i-1},0,\ldots,0)$. Since the input only depends on the first $i-1$ inputs, the normalizing flow is \textit{autoregressive} and the Jacobian is triangular.

The implementation in tensorflow uses \textit{bijectors} which implements a local diffeomorphism between a manifold $M$ and a target manifold $N$ (which are our parameter spaces), i.e. $\phi:M\rightarrow N$ such that $\phi$ is differentiable and injective. In tensorflow it has three operations, Forward, Inverse, and log\_deg\_jacobian, which are exactly the three we want. By constructing a bijector for each masked input, the full normalizing map can be constructed.

\subsection{Implementaion and Motivation Using a 2D Example}
To give a motivation, I will begin with a 2 dimensional example. In this example I used \textsc{cobaya} to generate samples from two distributions. One is a pure gaussian centered at $(1/2,1/2)$ with covariance $0.005 I$  ($I$ is the identity matrix). The other is a circle of radius 1 with points generated by a gaussian based on its distance from the circle. In other words, it is a gaussian centered at $x^2+y^2$ with a mean of $x^2+y^2=1$ and a standard deviation $0.02$.

The first step is to find the parater difference distribution. Wrap the samples in the shorter chain until it is the same length as the longer chain, then take the difference of the two chains. In this example we get the following.

Now we can do the normalizing flow. As a general rule-of-thumb, the network will consist $2d$ MAfs each of $2$ hidden layers with $2d$ hidden units each, with $d$ the dimension of the distribution.
This will generally give an expressive model without introducing error from the inability to train all of the models parameters in a reasonable time. 
These parameters are tunable to whatever the user will decide, and some experimentation with these parameters is needed to ensure the best results.
In addition, we allow the parameters to be arbitrarily permuted between each MAF since the resulting probability of shift should be independent of parameterization.

\subsection{Data Emulation and MCMC Sampling Analysis}

In practice the situation is more complicated. As described above, to sample from the posterior, we need to compute the likelihood and the prior at a given point. The likelihood is defined as the conditional probability $P(d|\theta)$, and in general to data space and parameter space are diffferent.
Thus we need a way to emulate data vectors for a given parameter vector.

In addition, although a markov chain explicitely only depends on the previous state, it does implicitely imply on its entire history of states. The chain only forgets its history after a certain number of steps, which is called the \textit{autocorrelation time} and is denoted $\tau$. To ensure the chain has a chance to explore the entire parameter space and generate a large enough set of independent samples, we need to determine $\tau$ for our chains.

\subsection{Tension Metrics}
In a previous DES paper, the tension metrics are the following:
\begin{enumerate}
    \item Bayesian evidence ratio given by
	\[ R = \frac{\mathcal{E}_{AB}}{\mathcal{E}_A\mathcal{E}_B} \]
	Where $A$ and $B$ are data sets. This method can be written many ways using bayes theorem, which will make it appearent that this metric depends heavily on the prior volume.
    \item Bayesian suspiciousness. This metric attempts to remove the dependence on the prior volume by defining the suspiciousness as 
	\[ \log S = \log R - \log I \]
	Of particular interest here is the new value $I$ which is the information ratio, which is defined in terms of the KL divergence.
	\[ \log I = \mathcal{D}_A + \mathcal{D}_B - \mathcal{D}_{AB} \]
	\[ \mathcal{D} = \int \mathcal{P} \log(\frac{\mathcal{P}}{\Pi}) \]
    \item The method of parameter difference $\Delta$ and $n_\sigma$.
    \item Parameter difference in update form. Suppose you have two data sets $A$ and $B$. The idea is to look at the difference in mean and covariance between data set $A$ and data set $A+B$.
	\[Q_{\mathrm{UDM}} = {(\mu_A - \mu_{A+B})}^T{(C_A-C_{A+B})}^{-1}(\mu_A - \mu_{A+B}) \]
	$Q_{\mathrm{UDM}}$ will be $\chi^2$ distributed with $\rank(C_A-C_{A+B})$ degrees of freedom.
    \item Goodness of fit degredation. This is similar to the previous, where it looks at how the goodness of fit of the model in data set $A$ degrades after adding data set $B$. We have
	\[ Q_{\mathrm{DMAP}} = 2\mathcal{L}_{A}(\hat{\theta}_A) + 2\mathcal{L}_B(\hat\theta_B) - 2 \mathcal{L}_{A+B}(\hat\theta_{A+B}) \]
	with $\hat{\theta}$ being the parameter vector that maximizes the posterior, the maximum a posteriori. Again $Q_{\mathrm{DMAP}}$ is $\chi^2$ distributed. 
\end{enumerate}

\subsection{Metric 1: Parameter Difference}
The idea behind this metric is simple: if two data sets largely agree, there difference posterior will be centered at 0, so the integral will be close to 0.
To actually compute the integral we use the normalizing flow to learn the posterior and perform MCMC inegration. The integration error is determined using the Clopper-Pearson interval on the binomial distribution.

\subsection{Metric 2: Eigentension}
This metric is interesting.
We start by diagonalizing the covariance matrix on one of our data sets.
Then we take the ratio of the variance in the prior to the variance in the posterior and apply an ad hoc cut to determine which eigenmodes are well-measured.
The idea is that we should not include poorly measured eigenmodes in our tension analysis because the difference is dominated by the prior rather than the data itself.
Lastly we project the other data set onto the eigenmodes and perform the parameter difference metric on only the well-measured eigenmodes.

(here is a good place to compare with metric 1)

\subsection{Metric 3: Parameter Difference in Update Form}
As discussed above, we can compute the parameter $Q_{\mathrm{UDM}}$ by
\begin{equation}
    Q_{\mathrm{UDM}} = {(\mu_A - \mu_{A+B})}^T{(C_A-C_{A+B})}^{-1}(\mu_A - \mu_{A+B}) 
\end{equation}
The difference of means is precisely the mean of the parameter difference distribution, and we are using the covariance $C_A+C_{A+B}$.
Thus it is clear $Q_{\mathrm{UDM}}$ is $\chi^2$ distributed with degrees of freedom given by $\rank(C_A-C_{A+B})$. 
It is clear, however, that this metric relies on the parameter difference to be gaussian distributed because of its reliance on $Q_{\mathrm{UDM}}$ being $\chi^2$ distributed. Despite this, we proceed anyway. With proper calibration this metric can be useful even for non-gaussian posteriors.

\subsection{Metric 4: Goodness of Fit Degradation}

\subsection{Normalizing Flow Analysis}
One advantage of normalizing flows is its effectiveness for smaller data sets. (talk about NF vs KDE when the data is available).



\bigskip
\hrule
\subsection{Fisher Analysis}
Informally, the fisher matrix is captures information from an ideal experiment. It is related to the inverse of the diagonalized covariance matrix.  Thus the fisher matrix can help us constrain cosmological parameters.

\bigskip
\hrule
\textbf{\large References}

\url{https://www.colorado.edu/amath/sites/default/files/attached-files/2_28_2018.pdf}

\url{https://si.biostat.washington.edu/sites/default/files/modules/Geyer-Introduction\%20to\%20markov\%20chain\%20Monte\%20Carlo_0.pdf}

\url{https://towardsdatascience.com/monte-carlo-markov-chain-mcmc-explained-94e3a6c8de11}

\url{https://www.sheffield.ac.uk/polopoly_fs/1.60510!/file/MCMC.pdf}
\hrule


\section{Fields}
\subsection{Multifield Dark Energy}
So I don't keep having to look at this
\begin{defn}
Given a semi-riemannian manifold $M$ with metric $g$, the christoffel symbols are given by
$$ \nabla_a\partial_b = \Gamma^c_{ab}\partial_c $$
$$ \partial_ag_{bc} + \partial_bg_{ca} - \partial_cg_{ab} = 2g_{dc}\Gamma^d_{ab} $$
\end{defn}
Consider a metric of the form diag$(-1,a^2(t))$, so its determinant is $-a^6$.
The action is
$$ S=\int d^4x \sqrt{-g} \left[ \frac{1}{2}M_{p}^2 R -\frac{1}{2}\gamma_{ab}\partial_\mu\phi^a\partial^\mu\phi^b - V(\phi)+\mathcal{L}_m \right] $$
I only want to describe a homogeneous background, so the field is only a function of time. The action becomes
$$ S=\int d^4x \sqrt{-g} \left[ \frac{1}{2}M_{p}^2 R -\frac{1}{2}\gamma_{ab}\dot\phi^a\dot\phi^b - V(\phi)+\mathcal{L}_m \right] $$
Varying the field gives
$$ \delta S = \int d^4x \sqrt{-g}\left[ - \frac{1}{2}\partial_a(\gamma_{bc})\dot\phi^b\dot\phi^c - \frac{1}{2}\gamma_{ab}\frac{d}{dt}(\delta\phi^a)\dot\phi^b - \partial_aV\delta\phi^a \right] $$
$$ \delta S = -\frac{1}{2}(\delta\phi^a)(\sqrt{-g}\gamma_{ab}\dot\phi^b + \int d^4x \sqrt{-g}\left[ -\frac{1}{2}\gamma(\nabla_a\partial_b,\partial_c)\dot\phi^b,\dot\phi^c -\frac{1}{2}\gamma(\partial_b,\nabla_a\partial_c)\dot\phi^b\dot\phi^c + 3\frac{\dot a}{a}\gamma_{ab}\delta\phi^a\dot\phi^b + \gamma_{ab}\delta\phi^a\ddot\phi^b -V_a\delta\phi^a \right] $$
$$ \delta S = -\int d^4x \sqrt{-g}\left[ \frac{1}{2}\Gamma_{ab}^d\gamma_{dc}\dot\phi^b\dot\phi^c + \frac{1}{2}\Gamma_{ac}^d\gamma_{bd}\dot\phi^b\dot\phi^c + 3H\gamma_{ab}\dot\phi^b + \gamma_{ab}\ddot\phi^b + V_a  \right] $$
Multiply everything by $\gamma^{aa}$. Now lets do some reshuffling of the indices.
$$ \gamma^{aa}\Gamma^d_{ab}\gamma_{dc}\dot\phi^b\dot\phi^c $$
$$ a\leftrightarrow d$$
$$ \gamma^{ad}\Gamma^a_{db}\gamma_{ac}\dot\phi^b\dot\phi^c $$
$$ b\leftrightarrow d$$
$$ \gamma^{ad}\Gamma^a_{bd}\gamma_{ac}\dot\phi^b\dot\phi^c $$
$$ \gamma^{ab}\Gamma^a_{bd}\gamma_{ac}\dot\phi^d\dot\phi^c $$
$$ c\leftrightarrow d$$
$$ \gamma^{ab}\Gamma^a_{bc}\gamma_{ad}\dot\phi^d\dot\phi^c = \Gamma_{bc}^a \dot\phi^b\dot\phi^c $$
and
$$ \gamma^{aa} \Gamma^d_{ac}\gamma_{bd}\dot\phi^b\dot\phi^c $$
$$ a\leftrightarrow d$$
$$ \gamma^{ad} \Gamma^a_{dc}\gamma_{ba}\dot\phi^b\dot\phi^c $$
$$ b\leftrightarrow d$$
$$ \gamma^{ab} \Gamma^a_{bc}\gamma_{da}\dot\phi^d\dot\phi^c = \Gamma_{bc}^a\dot\phi^b\dot\phi^c $$
Thus the equation of motion is found,
$$ \ddot\phi^a + \Gamma_{bc}^a\dot\phi^b\dot\phi^c +3H\dot\phi^a + V^a = 0 $$
$$ D_t\dot\phi^a+3H\dot\phi^a + V^a = 0 $$

The riemann curvature is
$$ R^\alpha_{\beta\gamma\delta} = \partial_\beta\Gamma^\alpha_{\gamma\delta} - \partial_\gamma\Gamma^\alpha_{\beta\delta}+\Gamma^\sigma_{\gamma\delta}\Gamma^\alpha_{\beta\sigma} - \Gamma^\sigma_{\beta\delta}\Gamma^\alpha_{\gamma\sigma} $$
So the Ricci curvature tensor is
$$ R_{\beta\delta} = R^\alpha_{\beta\alpha\delta} = \partial_\beta\Gamma^\alpha_{\alpha\delta} - \partial_\alpha\Gamma^\alpha_{\beta\delta}+\Gamma^\sigma_{\alpha\delta}\Gamma^\alpha_{\beta\sigma} - \Gamma^\sigma_{\beta\delta}\Gamma^\alpha_{\alpha\sigma} $$
And the Ricci scalar curvature is
$$ R = R_{\beta}^{\beta} = g^{\beta\delta}R_{\beta\delta} =
g^{\beta\delta}\partial_\beta\Gamma^\alpha_{\alpha\delta} -
g^{\beta\delta}\partial_\alpha\Gamma^\alpha_{\beta\delta}
+g^{\beta\delta}\Gamma^\sigma_{\alpha\delta}\Gamma^\alpha_{\beta\sigma} -
g^{\beta\delta}\Gamma^\sigma_{\beta\delta}\Gamma^\alpha_{\alpha\sigma} $$
Since the fields only depend on time, we only need to consider the temporal component of the Ricci curvature tensor, so
$$ R_{tt} = \partial_t \Gamma^{\alpha}_{\alpha t} - \partial_\alpha\Gamma^{\alpha}_{tt}
+ \Gamma^{\sigma}_{\alpha t}\Gamma^{\alpha}_{t \sigma} - \Gamma^{\sigma}_{tt}\Gamma^{\alpha}_{\alpha\sigma} $$
From the definition of the christoffel symbols, and noting $ g = g(t)$ and $g$ is diagonal, we have $\partial_t g_{aa} = 2g_{ba}\Gamma^b_{ta}$, thus
the second term is necessarily 0 since $\partial_t g_{tt} = 0$. The first term is non-zero for $\alpha = 1,2,3$, in which it equals 
$$ \frac{1}{2}\partial_t (1/a^2 \partial_t a^2) = \partial_t(\dot a / a) = -3\ddot a /a + 3\dot a^2 /a^2 $$
The third term is non-zero for $\alpha = \sigma$, In which case we get 
$$ \frac{3}{4a^4}(\partial_t (a^2))^2 =  -3\dot a^2/a^2  $$
The last term is vanishes since the temporal component of the metric is constant. Thus we find
$$ R_{tt} = 3\frac{\ddot a}{a}$$
Now, if we consider spatial components of the Ricci curvature, we find that the first term now vanishes, the second term is equal to $ -\ddot a/a - \dot a^2/a^2$, the thrid term is 0, and the last term is $-\dot a^2/a^2$, so the spacial components are
$$ R_{xx} = -\ddot a  a - 2\dot a^2$$
Thus the Ricci curvature scalar is
$$ 6\ddot a/a + 6\dot a^2/a^2 $$
Hence the einstein equation gives
$$ -3H^2 =  $$
I need to write out the indices better if I want to get the signs correct.

\hrule
$$ \partial_ag_{bc} + \partial_bg_{ca} - \partial_cg_{ab} = 2g_{dc}\Gamma^d_{ab} $$

\hrule
Considering the way this acts on the coordinate vector field $\partial_\alpha$, 
$$ R\partial_\alpha =
g^{\beta\delta}\partial_\beta\Gamma^\alpha_{\alpha\delta}\partial_\alpha -
g^{\beta\delta}\partial_\alpha\Gamma^\alpha_{\beta\delta}\partial_\alpha
+g^{\beta\delta}\Gamma^\sigma_{\alpha\delta}\Gamma^\alpha_{\beta\sigma}\partial_\alpha -
g^{\beta\delta}\Gamma^\sigma_{\beta\delta}\Gamma^\alpha_{\alpha\sigma} \partial_\alpha $$
$$ R\partial_\alpha =
g^{\beta\delta}\partial_\beta \nabla_\alpha\partial_\delta -
g^{\beta\delta}\partial_\alpha \nabla_\beta\partial_\delta
+g^{\beta\delta}\Gamma^\sigma_{\alpha\delta} \nabla_\beta\partial_\sigma -
g^{\beta\delta}\Gamma^\sigma_{\beta\delta} \nabla_\alpha \partial_\sigma $$

\end{document}
