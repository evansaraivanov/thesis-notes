\section{General Relativity}

To align with my interest I will write this section from a mathematical perspective and give a physics interpretation after. In this section, I will derive Einstein's field equation. First starting with defining smooth structures on a manifold. Afterwards I move into curvature before using the Bianchi identity on the Riemann tensor to derive the equation governing General Relativity. I will also discuss some properties of Einstein's Field Equation.
\subsection{Smooth Manifolds}
The basic structure of our universe according to general relativity is that the universe is a smooth 4-manifold. This already contains a lot of information so lets break it down.
To begin, one starts with defining a topological manifold. Given a space $X$ with topology $\tau$, $X$ is an $n$ dimensional topological manifold if
\begin{itemize}
	\item $X$ is Hausdorff. For each $p,q \in X$ there exists neighborhoods $p\in U$, $q\in V$, $U,V\in \tau$ such that $U \cap V = \empty$.
	\item $X$ is second countable. There exists a countable basis for the topology $\tau$.
	\item $X$ is locally Euclidean. For all $U\in\tau$ there exists a homemorphism $\phi:U\rightarrow V$ such that $V\subset \mathbb{R}^n$.
\end{itemize}
The last condition is one that hints at a differentiable structure becuase of known calculus in $\mathbb{R}^n$. To solidify this, consider the pair $(U,\phi)$ called a chart. Given two charts $(U,\phi)$ and $(V,\psi)$ with $U\cap V \neq \empty$, we can create the transition map $\psi \circ \phi^{-1}:\phi(U\cap V) \rightarrow \psi(U\cap V)$. The two charts are smoothly compatible if $U\cap V= \emptyset $ or the transition map is a homeomorphism.

Manifolds are a generalization of surfaces to arbitrary dimension. To make this a true generalization one needs a way to compute derivatives on a manifold. Let $M$ be a manifold with $x\in M$ and let $\gamma:(-\epsilon,\epsilon)\rightarrow M$ be a curve on $M$ with $\gamma(0)=x$. A tangent vector of $M$ is the equivalence class of curves, $[\gamma(t)]$, with equal $\gamma'(0)$. The tangent space of $M$ at $x$, denoted $T_xM$ is the set of all tangent vectors at $x$. The tangent bundle of $M$ is the disjoint union of all tangent spaces of $M$, denonted $TM = \bigsqcup\limits_{x\in M} T_xM$. The tangent bundle is constructed as a trivial bundle, so one can easily define $(T_xM,x)$ as fibers of the bundle and a projection $\pi:TM \rightarrow M$ with $\pi(T_xM,x)\rightarrow x$. Thus sections take $x$ to the tangent space at $x$.

With this structure, one can now infinitessimaly take a vector in a tangent space and transport it to an adjacent tangent space, thus giving rise to the differential structure seeked in the previous paragraph. Given a vector field $v$ on $M$, $D_v$ is a connection on $M$ if the usual derivative properties hold (linearity over $C^{\infty}(M)$ and Leibniz's law). 

Usual definition of vector field: a function $v:C^\infty(M) \rightarrow C^\infty(M)$

\subsection{Curvature}
Suppose one have a (semi-) Riemannian manifold $M$ with metric $g$ and tangent bundle $TM$. An \textit{affine connection} is a map
\begin{equation}
	\nabla : \Gamma (TM) \times \Gamma (TM) \rightarrow \Gamma (TM)
\end{equation}
\begin{equation}
	(X,Y) \mapsto \nabla_X Y
\end{equation}
That is, it parallel transports the vector field $Y$ along the connection $\nabla$ in the direction of vector field $X$. From this, one can write the affine geodesic equation for a path $\gamma(t)$
\begin{equation}
	\nabla_{\dot\gamma} \dot\gamma(t) = 0
\end{equation}
Thus, a geodesic is a path such that its tangent vector is parallel translated. Since one observes their world with coordinates, in physics it is often more instructive to work this out in a specific set of coordinates $x^\mu$. Thus this can be written as
\begin{equation}
	\ddot{\gamma}^\mu + \Gamma^{\mu}_{\rho\lambda}\dot{\gamma}^\rho \dot{\gamma}^\lambda 
\end{equation}

In general, affine connections have three components: a curvature component, a torsion component, and a \textit{metricity} component. 

\subsection{Einstein's Field Equation}
In a curved spacetime, conservation of energy is written
\begin{equation}
	\nabla^\mu T_{\mu\nu} = 0
\end{equation}
When deriving Einstein's equation, one wants to find a divergence-less tensor that depends only on the geometry. The following procedure follows closely one would to find the classical Yang-Mills equation for gauge fields. The Riemann curvature is anti-symmetric in the first two lower indices, so the Riemann tensor is like a GL(TM) valued differential 2 form, and the bianchi identity is
\begin{equation}
	d_{\nabla} R = 0
\end{equation}
Explicitely writing this becomes
\[ \nabla_{\alpha} R^{\alpha}_{\beta\gamma\delta} + \nabla_{\beta}R^{\alpha}_{\gamma\alpha\delta} + \nabla_{\gamma}R^{\alpha}_{\alpha\beta\delta} = 0 \]
\[ \nabla_{\alpha} R^{\alpha}_{\beta\gamma\delta} + \nabla_{\beta}R^{\alpha}_{\gamma\alpha\delta} - \nabla_{\gamma}R^{\alpha}_{\beta\alpha\delta} \]
\[ \nabla_{\alpha}R^{\alpha}_{\beta\gamma\delta} + \nabla_{\beta}R_{\gamma\delta} - \nabla_{\gamma}R_{\alpha\delta} \]
Multiplying by $g^{\beta\delta}$ and doing some relabelling/contractions of internal indices one finds
\begin{equation}
	\nabla^\alpha(R_{\gamma\alpha}-\frac{1}{2}g_{\gamma\alpha}R) \equiv \nabla^{\alpha}G_{\gamma\alpha} = 0
\end{equation}
There is one more divergencless tensor, the metric tensor $g_{\mu\nu}$, so one can write
\begin{equation}
	G_{\mu\nu} - \Lambda g_{\mu\nu} = 8\pi\kappa T_{\mu\nu}
\end{equation}

\section{$\Lambda$CDM: The Standard Model of Cosmology}


\subsection{The FLRW Metric and Dark Energy}
In general, metrics allow one to attribute distances between points in a space as $d = g_{\mu\nu}x^\mu x^\nu$. 
The `flat' metric for relativity is the \textit{Minkowski metric} given by $\text{diag}(-1,1,1,1)$, however, a metric to describe the expanding universe is given by the \textit{FLRW metric} by $d^2 = t^2-a^2(t)s^2$ with $s^2$ the standard Euclidean distance in $\mathbb{R}^3$. 
What is immediately appearent from the FLRW metric is that spacial slices of the $d=4$ spacetime remain curvature free under the expansion of the universe describe by the scale factor $a^2(t)$.

In cosmology, there are other distances which can be more useful than the distance given by the FLRW metric. 
In the FLRW metric the distance between two points grows in time. 
We can avoid this by defining the \textit{comoving distance} in which distances remain fixed through time. 
If we look at a coordinate function $x^\mu$ at $t=t_0$, at a later time the coordinate function can be written as $x^\mu \rightarrow a(t) x^\mu$, thus by dividing by the scale factor $a(t)$ we can define the comoving coordinates as

\begin{equation}
    \chi = \int\limits^{t}_{t_0} \frac{1}{a(t')} dt'
\end{equation}

with the standard Minkowski metric. This can be taken a step further by determining how far light has travelled since $t=0$
\begin{equation}
    \eta = \int\limits^t_0 \frac{1}{a(t')}dt'
\end{equation}

Since we can't see anything beyond this distance, it is often called the \textit{comoving horizon}. 
There is one last useful distance to define, the \textit{angular distance} which is inferred by the angle subtended by two objects. 
This relates distances to the geometry discuss in the first section where the measured distance will be the radial distance $D$

\begin{equation}
    D_A = \left\{ \begin{array}{cc}
	    R & K=0 \\
	    R\sin(D/R) & K>0 \\
	    R\sinh(D/R) & K<0
    \end{array}
    \right.
\end{equation}

When describing the structure of the universe, I will make a few assumptions (which hold up to small perturbations):

\begin{itemize}
    \item Homogeneity. The cosmology describing the universe does not depend on location.
    \item Isotropy. The cosmology describing the universe does not depend on direction.
\end{itemize}

These two conditions for what is sometimes referred to as \textit{the cosmological principle}. 
In general, they don't hold on small scales, however averaging over a sufficiently large distance these assumptions give an accurate description. 

If we examine the $00$ component of Einstein's Field Equation, the result is the \textit{first Friedman equation} (note that $\rho$ is shorthand for $\sum_i\rho_i$ and $R(G)$ to note that $R$ depends on the geometry of spaceial slices)

\begin{equation}
    H^2(a) + \frac{1}{a^2 R^2(G)} = \frac{8\pi G}{3}\rho
\end{equation}

We can interpret the second term on the left (which is spacial curvature) as some density asociated with the curvature $\rho_k$. If we divide by $\rho_{\text{crit}}$ at $z=0$ / $a=1$ we find the usual form. 

\begin{equation}
    \omega + \omega_k = 1 
\end{equation}

The second Friedman equation comes from the trace of Einstein's equation.

\begin{equation}
    \frac{\ddot a}{a} = -\frac{4\pi G}{3}(\rho + 3P)
\end{equation}

\subsection{Perturbations of the FLRW Metric}
Upon looking at the sky, one should realize that our universe is not homogenous. We can see clusters of stars and galaxies as well as regions devoid of matter. Thus, the FLRW metric is taken as a baseline approximation which we perturb in order to correct for our observations. In general relativity, the Christoffel symbols are the gravitational potential generating the gravitational field. The definition of the Christoffel symbols in terms of the metric is

\begin{equation}
	\Gamma_{\mu\nu\lambda} = \frac{1}{2} 
	\left(
		\partial_\lambda g_{\mu\nu} + \partial_{\nu} g_{\mu\lambda} - \partial_{\mu}g_{\nu\lambda} 
	\right)\,.
\end{equation}

The FLRW metric only has one non-zero partial derivative, the derivative with respect to $t$. Thus we if $\lambda=0$ or $\nu=0$, we have

\begin{equation}
	\begin{split}
		\Gamma_{ij0} = \Gamma_{i0j} = \delta_{ij}a \dot a \\
		\Gamma_{j0}^i = \delta_{ij} \frac{\dot a}{a} \\
		\Gamma_{ij}^0 = \delta_{ij} a \dot a
	\end{split}
\end{equation}

Thus, with two unique Christoffel symbol values, we can perturb the metric using two potentials $\Phi$ and $\Psi$. So the perturbed metric becomes 

\begin{equation}
	d^2 = t^2(1+2\Psi) - a^2(t)s^2(1-2\Phi)
\end{equation}

where the perturbations are much smaller than $1$. For the remainder of this thesis, calculations involving the geometry of the universe will be based on this perturbed metric.
















