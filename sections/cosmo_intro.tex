\chapter{Cosmology}
\section{Introduction}
With Hubble's discovery of the expanding universe, there has been great efforts to understand this expansion and the evolution of this expansion (and of the universe as a whole).
A related idea, General Relativity has described the importance of the geometry of the universe.
There are, in general, three possible geometries:
\begin{itemize}
    \item A flat geometry which is equivalent to Euclidean space with zero curvature.
    \item An open geometry with constant negative curvature (Anti-de Sitter space).
    \item A Closed geometry with constant positive curvature (de Sitter space)
\end{itemize}
Given the relationship between curvature and the energy-momentum tensor given by the Einstein field equations, it seems reasonable to assume that any non-zero energy means the universe is not flat.
Through quantum field theory (QFT) the Casimir effect shows that the vacuum has non-zero energy density, and through astronomical observation of distant galaxies it can be seen that our universe is flat.
At first these two observations contradict each other. However, by introducing another term to the Einstein field equations, this discrepency can be resolved.
\begin{equation}
    G_{\mu\nu} - \underbrace{\Lambda g_{\mu\nu}}_{\text{new term}} = T_{\mu\nu}
\end{equation}
The constant $\Lambda$ is called the \textit{cosmological constant} which can absorb the contributions from the vacuum energy, allowing for a flat universe and non-zero vacuum energy density.
A natural question to ask is `what is the source of the cosmological constant?'